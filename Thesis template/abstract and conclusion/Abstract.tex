%ABSTRACT
Wireless Sensor Networks (WSNs) is a sub-class of wireless Ad-hoc networks where sensor nodes are randomly deployed to gather some specific type of information about physical phenomena such as temperature, pressure, humidity, rainfall, etc. The nodes in an ad-hoc network are small battery-operated devices with limited computational capabilities and transceivers to communicate with nearby nodes and the base station (BS) or a gateway to another network. However, in WSNs, these nodes also have additional sensor/transducer modules to measure some desired physical phenomenons, hence, called sensor nodes or sensors. The lifetime of a WSN can be defined as the functional duration of nodes in WSNs. The lifetime of the network (or nodes in the network) depends on the initially installed battery and power consumption of the nodes. To prolong the lifetime of network operation like routing, sensing, communications protocols are planned differently.
...


