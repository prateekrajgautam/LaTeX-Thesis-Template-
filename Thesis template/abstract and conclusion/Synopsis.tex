%ABSTRACT
Wireless Sensor Networks (WSNs) is a sub-class of wireless Ad-hoc networks where sensor nodes are randomly deployed to gather some specific type of information about physical phenomena such as temperature, pressure, humidity, rainfall, etc. The nodes in an ad-hoc network are small battery-operated devices with limited computational capabilities and transceivers to communicate with nearby nodes and the base station (BS) or a gateway to another network. However, in WSNs, these nodes also have additional sensor/transducer modules to measure some desired physical phenomenons, hence, called sensor nodes or sensors. The lifetime of a WSN can be defined as the functional duration of nodes in WSNs. The lifetime of the network (or nodes in the network) depends on the initially installed battery and power consumption of the nodes. To prolong the lifetime of network operation like routing, sensing, communications protocols are planned differently.
\par
WSNs are a network of a large number of sensor nodes deployed to collect information for various applications, such as monitoring and surveillance, disaster relief, health care, home and industrial automation. This information gathered by sensors is transmitted to the base station for processing and control operations. However, the information at the base station is useful only if the location of the sensor node is known. For example, if a WSN is deployed to detect fire in a region and in a case of fire, a sensor communicates this to the base station, so in the absence of location information, we only know that there is a fire somewhere in the region. So, we need to search the entire region to detect the location of the fire. This search in a crucial moment can be avoided if the sensor node knows its location. Further, the location of sensor nodes is often used to optimize location-based routing protocols, thereby, reducing energy consumption. The process of finding the location of randomly deployed sensor nodes in the WSNs is called localization and it is a crucial task in WSNs. Further, as per the energy constraints in WSNs, the localization process in WSNs must be energy efficient and its computational complexity should be small.
\par
The architecture of WSNs mainly consists of sensor nodes and a base station. The nodes gather physical information and form network to communicate this information to the base station. The information is processed centrally at the base station. There are two types of sensor nodes in WSNs, anchor (or beacon) node and target (or dumb) node. Nodes that know their location either via GPS or manual deployment are used to assist in localization are called anchor nodes, anchors, or beacon nodes. The nodes that do not know their location are called target nodes or dumb nodes. The localization process aims to find the location of these target nodes.
%
\par The main objection of this thesis is to present low-complexity energy-efficient localization techniques capable to localize using single anchor nodes or GPS-enabled nodes in the WSN. The localization techniques presented are scalable, fast, and easily implementable at the nodes, and work for homogeneous as well as heterogeneous networks. The nodes act in passive listening mode, so their energy consumption is reduced in comparison to existing localization processes. Three energy-efficient localization techniques titled are presented each capable of localizing nodes with a single anchor node.
The title of the proposed schemes are:
\begin{enumerate}
\item \emph{Energy-efficient Localization of sensor nodes in WSNs using Single Beacon node (EE-LSB)}. \item \emph{Energy-efficient Localization of sensor nodes in WSNs using Beacons from Rotating Directional antenna (EE-LBRD)}.
\item \emph{Three-dimentional Localization scheme using Beacons from Rotating Directional antennas (3D-LBRD)}.
\end{enumerate}
EE-LBRD and 3D-LBRD are independent of received signal strength indicator (RSSI) based distance estimation. 3D-LBRD is capable of three-dimensional (3D) localization of nodes while EE-LSB and EE-LBRD are 2D localization.
%%EE-LSB
\par EE-LSB is low complexity, energy-efficient, and cost effective localization process which does not require any hardware modification at the node side. The anchor node uses a wide beamwidth directional antenna (DA) to transmit beacon packets. Anchor node transmits two types of beacons, one from an Omni-directional antenna for radial distance estimation and another from the directional antenna for direction estimation. The Omni-directional antenna of the anchor node transmits several beacons each with different transmit power so its communication range is changed. The directional antenna of the anchor node has a small beam-width and it transmits beacons containing the current direction of the directional antenna. The direction of the directional antenna is changed after each transmission. Nodes received this information and estimate their location individually. Localization is performed in polar coordinates and the final location depends on distance estimated using RSSI. The total number of beacons transmitted from the single beacon node (SBN) is 56. Thus, energy consumption during the localization is reduced at the node side as well as at the anchor node because the number of transmissions from the anchor node is reduced and no transmission is required from nodes. This scheme is validated analytically and by simulated by MATLAB for hundred randomly deployed nodes in $100$m $\times$ $100$m two dimensional (2D) plane. Further, the total time required for the localization of all the nodes in the region is 5.6 seconds and the average localization error is 1.93m.%comparision with...
%%EELBRD
\par EE-LBRD uses a rotating directional antenna (RDA) at the anchor node to transmit a beacon signal packet containing information about the height of the antenna and current direction.
The direction of the directional antenna can be changed on two axis to change its elevation angle and azimuth angle. Each node estimates its location relative to the anchor node in polar coordinate using the information received from the beacon packets. The time required, energy consumption, and accuracy are estimated analytically. Simulation of the proposed scheme is performed to study the localization error, effect of change of parameters on the accuracy, time required to complete localization, and energy consumption of the node. MATLAB programming is used for simulation a scenario of hundred randomly deployed nodes in a $100$m $\times$ $100$m 2D plane shows that the proposed scheme improves accuracy and energy-efficiency for the two-dimensional localization of sensor nodes in WSNs. Localization is performed with an average accuracy of $\pm1.5$m, the energy consumption of any node is less than 70mJ.%, and the energy efficiency of localization is 5.4.
%%3D-LBRD
\par 3D-LBRD is a three-dimensional (3D) localization technique based on EE-LBRD. The anchor node transmits beacons using its two rotating directional antennas fixed at different heights. Each node receives the beacons from directional antennas and estimates their location. The 3D locations of nodes are estimated analytically via graphical representation. The scheme is verified with MATLAB simulation where a hundred nodes are randomly deployed in a 3D region ($100$m $\times$ $100$m $\times$ $20$m) and validated with a hardware setup experimentally. The directional antenna for experimental setup is designed and optimized using CST Studio Suite using a conducting cavity structure to attain a directivity of 9.43 and beamwidth of $60^\circ$. The experimental results prove the concept, and the results obtained are close to simulation results. The average error is less than $1.1$m. Further, an application of the proposed work is presented to monitor inventory levels in the industrial model.%final concluding remark
\par The main contributions of the thesis can be summarized as follows. It present low complexity energy-efficient 2D and 3D localization schemes in WSNs that do not require any hardware modification at the node side. The localization schemes proposed to achieve accuracy better than existing schemes without increasing energy consumption. The schemes require a single anchor node, hence, its impact on the cost of localization is small and the schemes can be easily scaled. The computational complexity of schemes is low and can be implemented in a distributed manner on individual nodes without the need for additional hardware. Implementation of schemes is demonstrated with the experimental setup, The design of anchor node and directional antenna required for the proposed scheme is presented with its experimental model. Finally, the application of the proposed 3D localization scheme is simulated for industrial inventory monitoring.
\par In the future, the accuracy of 3D localization can be improved using intercommunication among nodes and optimization. The scheme can be modified for the localization of mobile nodes. 3D-LBRD can be extended for underwater localization.



